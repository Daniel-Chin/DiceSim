\documentclass[12pt]{article}
\usepackage{titling}
\usepackage{amsmath}
\usepackage{amssymb,graphics,color,cite,amsmath}
\usepackage{setspace}
\usepackage{graphicx}
\usepackage{subfigure}
\usepackage{float}
\usepackage{url}
\usepackage[margin=0.75in]{geometry}

\title{Simulating Roll of a Dice Using Rigid Body Motion}
\date{2019-10}
\author{Daniel Chin and Michael Li}
\pagenumbering{arabic}
\scriptsize

\begin{document}

\maketitle
\section{Introduction}
In this project, we use the rigid body motion model to simulate the roll of a dice. The dice is represented by 8 mass points at corners of the cube. 
\section{Mathematical Modeling}
\subsection{Rigid Body Motion}
The total mass and center of mass of the system, as well as the velocity of the center of mass are defined as follows:\\
\begin{equation}
M=\sum_{k=1}^{n}M_{k}
\end{equation}
\begin{equation}
M\vec{X_{cm}}=\sum_{k=1}^{n}{M_{k}{\vec{X_{k}}}}
\end{equation}
\begin{equation}
M\vec{U_{cm}}=\sum_{k=1}^{n}{M_{k}{\vec{U_{k}}}}
\end{equation}\\
Let $\vec{L}$ be the angular momentum of the system about its center of mass:\\
\begin{equation}
\aligned
\vec{L(t)}&=\sum_{k=1}^{n}M_{k}(\vec{X_{k}}-\vec{X_{cm}})\times \vec{U_{k}}\\
&={\sum_{k=1}^{n}M_{k}(\vec{X_{k}}-\vec{X_{cm}})\times (\vec{U_{k}}-\vec{U_{cm}})}
\endaligned
\end{equation}\\
Differentiate both sides with respect to $t$ and use the identity $(\vec{U_{k}}-\vec{U_{cm}})\times (\vec{U_{k}}-\vec{U_{cm}})=0$\\
\begin{equation}
\aligned
\frac{d{\vec{L}}}{dt}&=\sum_{k=1}^{n}(\vec{X_{k}}-\vec{X_{cm}})\times (M_{k}\frac{d{\vec{U_{k}}}}{dt}-M_{k}\frac{d{\vec{U_{cm}}}}{dt})\\
&=\sum_{k=1}^{n}(\vec{X_{k}}-\vec{X_{cm}})\times \vec{F_{k}}\\
&=\sum_{k=1}^{n}\vec{\tilde{X_{k}}}\times \vec{F_{k}}\\
&=\vec{\tau}
\endaligned
\end{equation}
where $\vec{F_{k}}$ denotes the external force acting on a node, $\vec{\tilde{X_{k}}}$ is the positional vector of this node, and $\vec{\tau}$ is the total torque of the system.\\
\begin{equation}
M\frac{d{\vec{U_{cm}}}}{dt}=\vec{F}=\sum_{k=1}^{n}{F_{k}}
\end{equation}
Since we want the motion to be rigid, we should solve for the angular velocity. So we shall introduce the individual point velocity $\vec{u_{k}(t)}$:\\
\begin{equation}
\vec{u_{k}(t)}=\vec{u_{cm}(t)}+\vec{\Omega}\times(\vec{X_{k}(t)}-\vec{X_{cm}(t)})
\end{equation}
Then we plug the above results into the previous definition of torque:\\
\begin{equation}
\aligned
\vec{L(t)}&=\sum_{k=1}^{n}M_{k}(\vec{X_{k}(t)}-\vec{X_{cm}(t)})\times (\vec{\Omega(t)}\times (\vec{X_{k}(t)}-\vec{X_{cm}(t)}\\
&=\sum_{k=1}^{n}M_{k}{\lVert{\vec{X_{k}(t)}-\vec{X_{cm}(t)}}\rVert}^{2}\vec{\Omega(t)}-\sum_{k=1}^{n}M_{k}(\vec{X_{k}(t)}-\vec{X_{cm}(t)})^{2}\cdot \vec{\Omega}\\
&=\vec{I(t)}\vec{\Omega(t)}
\endaligned
\end{equation}\\
where $\vec{I(t)}=\sum_{k=1}^{n}M_{k}({\lVert{\tilde{\vec{X_{k}(t)}}}\rVert}^{2}E_{3\times3}-\tilde{\vec{X_{k}(t)}}(\tilde{\vec{X_{k}(t)}})^{T})$ is a $3\times3$ matrix, denoting the moment of inertia.\\
We can solve for $\vec{\Omega}$ by computing \\
\begin{equation}
\vec{\Omega}={\vec{I(t)}}^{-1}\vec{L(t)}
\end{equation}
Once we know $\vec{\Omega}$, we construct the rotation matrix $R(\vec{\Omega},\Delta{t})$. In particular, $R$ rotates every position vector about the rotating axis for an angle $\vec{\Omega(t)}\delta{t}$.\\
Then we construct the projection matrix onto the unit vector in the direction of $\vec{\Omega}$, which is rotational-invariant.\\
\begin{equation}
P(\vec{\Omega})=\frac{\vec{\Omega}}{\lVert{\vec{\Omega}}\rVert}(\frac{\vec{\Omega}}{\lVert{\vec{\Omega}}\rVert})^{T}
\end{equation}
We express $\tilde{\vec{X}}$ as the following:
\begin{equation}
\tilde{\vec{X}}=P(\vec{\Omega})\tilde{\vec{X}}+(E_{3\times3}-P(\vec{\Omega}))\tilde{\vec{X}}
\end{equation}
The former ingredient is rotational invariant, while the latter ingredient is rotational about $\lVert{\vec{\Omega}}\rVert\Delta{t}$ through the origin, normal to $\vec{\Omega}$.
Finally, by taking the orthonormal basis $(\vec{v_1},\vec{v_2})$, where $\vec{v_1}=(E_{3\times3}-P(\vec{\Omega})\tilde{\vec{X}}$, $\vec{v_2}=\frac{\vec{\Omega}}{\lVert{\vec{\Omega}}\rVert}\times\vec{v_1}$,we achieve the orthogonal matrix in terms of $\tilde{\vec{X}}$ and $\lVert{\vec{\Omega}}\rVert\Delta{t}$:\\
\begin{equation}
R(\vec{\Omega},\Delta{t})\tilde{\vec{X}}=P(\vec{\Omega})\tilde{\vec{X}}+cos(\lVert{\vec{\Omega}}\rVert\Delta{t})(E_{3\times3}-P(\vec{\Omega}))\tilde{\vec{X}}+sin(\lVert{\vec{\Omega}}\rVert\Delta{t})\frac{\vec{\Omega}}{\lVert{\vec{\Omega}}\rVert}\times\tilde{\vec{X}}
\end{equation}
\subsection{Gravity and Interaction with the Ground}
\begin{itemize}
  \item \textbf{Gravity:} $G=mg$
  \item \textbf{Normal Force:} The ground is modeled a stiff trampoline in 3-D space. Express the ground surface as $H$.Mass points receive an upward normal force proportional to how deep it penetrates the ground:\\
\begin{equation}
F_{normal}=-min(0,X_{z})\cdot k
\end{equation}
       where $k=500000 N/cm$ is the ground stiffness,and where $X_{z}$ is the vertical component of the position of mass point.\\
  \item \textbf{Damping force:} The collision of the dice with the ground transfers some mechanical energy to heat. A damping force is added whenever the mass point is below $h=0$:\\
\begin{equation}
F_{damping}=-F_{normal}\cdot u_{h}\cdot \gamma
\end{equation}
where the damping coeffient $\gamma=0.005 s/cm$, $u_{z}$ is the vertical component of the velocity of the mass point.\\
  \item \textbf{Friction:} 
\begin{equation}
F_{friction}=-|F_{normal}+F_{damping}|\cdot \mu \cdot \frac{u_{x,y}}{\lVert{u_{x,y}}\rVert}
\end{equation}
where $\mu=1.2$ is the frictional coefficient. However, we do not model static friction since we end the simulation before the dice completely stabilizes. 
  \item \textbf{Adjusted Moment of Inertia:} If we compute the moment of inertia $\vec{I(t)}$ from the 8 mass points as discussed above, we will overestimate, since a real dice has uniform concentration of mass throughout the cube. Fortunately, we can simply replace the value of I in the simulation:\\
\begin{equation}
\vec{I(t)}=\frac{1}{6}ms^{2}=\frac{16}{3}\times E_{3\times3}
\end{equation}
Since the matrix is invariant to rotation, we do not need to update $\vec{I(t)}$ during the simulation. 
  \item \textbf{Other Constants:} 
\begin{itemize}
  \item The radius of the dice is $1 cm$. 
  \item The initial height of the dice is $15 cm$. 
  \item The time step is $\Delta{t}=0.001 sec$. 
\end{itemize}
\end{itemize}
\section{Numerical Methods}
We use forward Euler's method to perform a timestep, given $\vec{X_{k}(t)}$, $\vec{X_{cm}(t)}$, $\tilde{\vec{X_{k}(t)}}$, $\vec{U(t)}$, $\vec{U_{cm}(t)}$, and $\vec{L(t)}$:\\
\begin{itemize}
  \item Given $\tilde{\vec{X_{k}(t)}}$, compute $I(t)$. Note that in our model, $I(t)$ does not vary with time.
  \item Solve for $\vec\Omega(t)$.
  \item Using $\vec\Omega(t)$, update $\tilde{\vec{X_{k}(t+\Delta{t})}}=R(\vec{\Omega(t)},\Delta{t})\tilde{\vec{X_{k}(t)}}$.
  \item Update the center of mass: $\vec{X_{cm}(t+\Delta{t})}=\vec{X_{cm}(t)}+\Delta{t}\vec{U_{cm}(t)}$.
  \item Update the velocity of the center of mass: 
\begin{center}
$M\frac{\vec{U_{cm}(t+\Delta{t})}-\vec{U_{cm}(t)}}{\Delta{t}}=\sum_{k}\vec{F_{k}(t+\Delta{t})}$
\end{center}
  \item Update the positions of the individual masses: 
\begin{center}
$\vec{X_{k}(t+\Delta{t})}=\vec{X_{cm}(t+\Delta{t})}+\tilde{\vec{X_{k}(t+\Delta{t})}}$
\end{center}
  \item Update the angular momentum:
\begin{center}
$\frac{\vec{L(t+\Delta{t})}-\vec{L(t)}}{\Delta{t}}=\sum_{k}\tilde{\vec{X_{k}(t+\Delta{t})}}1\times\vec{F_{k}(t+\Delta{t})}$
\end{center}
\end{itemize}
\section{Simulation Results}
\subsection{Assumptions and Cautionaries}
\begin{itemize}
  \item \textbf{Rigid Body:}\\
If you drop a basketball with a backward spin, it is likely to bounce up with a forward spin. That is because the ball stores energy as rotational elastic deformation as it hits the ground. Since we model the dice as a rigid body, we will lose this effect. However, this problem can be ignored if we assume the elasticity of the table (“ground stiffness”) is the dominant factor in the interaction. 
  \item \textbf{Perfect Ground and Air Drag:}\\
We do not model for ground deformation or air drag. 
  \item \textbf{Super Sharp Corner:}\\
Because the corners of the dice are infinitely sharp right angles, the dice will not trigger the precession effect. Real dices can spin rapidly on their corner for a long time because the precession effect helps them self-stabilize. This is a major deviation of our model from reality. 
\end{itemize}
The potential energy of the dice when it is standing on its edge is $mg\sqrt{2}=0.0032474$. This is the critical threshold for the dice to flip over and change face. (Elastic potential energy is neglected here)
TThe simulation is ended preemptively whenever the total energy in the system is lower than $80\%$ of the above threshold. This is to reduce simulation time.\\
To decide which face is facing upwards, the program takes the highest four points and look up which face it is.\\

\subsection{Main Results}
Our physic model is deterministic:\\
\begin{equation}
\aligned
&y=f(ic)\\
&ic\in IC\\
&y \in Z\bigcap\left[{1,6}\right]
\endaligned
\end{equation}


\section{Further Questions}
\section{Acknowledgement}
The authors would like to thank Professor Peskin for his help and suggestions. Relevant documents and codes can be found here: \url{https://github.com/Daniel-Chin/DiceSim}.
\section{References}
\begin{itemize}
  \item Notes on Structural Dynamics with Stiff Links: The Backward-Euler Method, by Prof. C. Peskin. \url{https://www.math.nyu.edu/faculty/peskin/modsim_lecture_notes/backward_euler.pdf}\\
  \item Notes on Interactions of Dynamic Structures with the Ground: Non--Penetration and Slicing Frictional Forces, by Prof. C. Peskin. \url{https://www.math.nyu.edu/faculty/peskin/modsim_lecture_notes/interaction_with_ground.pdf}\\
  \item Notes on Dynamics of Rigid Bodies, by Prof. C. Peskin. \url{https://www.math.nyu.edu/faculty/peskin/modsim_lecture_notes/rigid_body_motion.pdf}\\
  \item Notes on Structural Dynamics with Rigid Links, by Prof. C. Peskin. \url{https://www.math.nyu.edu/faculty/peskin/modsim_lecture_notes/mechanics_with_rigid_links.pdf}\\
\end{itemize}

\end{document}
